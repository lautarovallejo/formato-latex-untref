\documentclass[12pt,a4paper]{article} %hoja A4 y tamño de letra 12pt
\usepackage{setspace}
\onehalfspacing %interlineado 1,5 (no parece el mismo de Word pero es Word el que hace mal eso, este es el posta)
\usepackage[spanish,es-tabla]{babel} %idioma español y "tabla" en lugar de "cuadro"
\usepackage{fontspec}
\setmainfont{Calibri}
%\setmainfont{Times New Roman} %solamente para la primera entrega
%\newfontfamily\myfont{Calibri} %solamente para la primera entrega
\usepackage{datetime}
\usepackage{fancyhdr} %para alinear números de página a la derecha
\usepackage{lipsum}
\renewcommand{\headrulewidth}{0pt}%
\newdateformat{monthyeardate}{
  \monthnamespanish[\THEMONTH]  \THEYEAR}  \makeatletter	
\renewcommand{\monthnamespanish}[1][\month]{%
  \@orgargctr=#1\relax
  \ifcase\@orgargctr
    \PackageError{datetime}{Invalid Month number \the\@orgargctr}{%
      Month numbers should go from 1 to 12}%
    \or Enero%
    \or Febrero%
    \or Marzo%
    \or Abril%
    \or Mayo%
    \or Junio%
    \or Julio%
    \or Agosto%
    \or Septiembre%
    \or Octubre%
    \or Noviembre%
    \or Diciembre%
    \else \PackageError{datetime}{Invalid Month number \the\@orgargctr}{%
      Month numbers should go from 1 to 12}%
  \fi}
\makeatother
\usepackage{amsmath,amsfonts,amssymb}
\usepackage{graphicx} %para figuras
\usepackage[font=footnotesize]{caption} %descripciones de tablas/figuras en 10pt
\graphicspath{ {imagenes/} } %carpeta donde guardar imágenes
\usepackage{luatodonotes}
\setlength{\marginparwidth}{2 cm}
\usepackage[left=3cm,right=2.5cm,top=2.5cm,bottom=2.5cm]{geometry} %márgenes
%\usepackage[sorting=none,maxbibnames=99,isbn=false,doi=false,eprint=false,giveninits=true]{biblatex}
\usepackage[
backend=biber,
sorting=none,
style=numeric-comp
]{biblatex} %referencias
\renewbibmacro{in:}{}
\DeclareNameAlias{default}{last-first}
\addbibresource{referencias.bib} %añade archivo de referencias
\usepackage{tikz}
\usetikzlibrary{calc}
\usetikzlibrary{babel}
\usepackage{titlesec}
\titleformat{\section}{\normalfont\fontsize{16}{19}\bfseries}{\thesection.}{0.5em}{\MakeUppercase} %formato de título
\titleformat{\subsection}{\normalfont\fontsize{14}{16}\bfseries}{\thesubsection.}{0.5em}{\MakeUppercase} %formato de subtítulo
\titleformat{\subsubsection}{\normalfont\fontsize{14}{16}}{\thesubsection.}{0.5em}{} %formato de sub-subtítulo
\let\labelitemi\labelitemii

%Esto se modifica como quieran:
\author{Autor}
\title{Título}
